\hoofdstuk{Conclusion and Recommendation}
% With Appcelerator Titanium one is able to develop cross-platform mobile applications while retaining the native look at feel.

% There are serveral solutions to crossplatform mobile development. However due to the high bar set to what Lunatech defines as native only Titanium is a possibility.


%Samenvatten wat er gedaan is aan de Stager app, welke platforms het draaid, e.d.
%TODO diagram met features en implementatie compleetheid
%Currently there are several solutions available to solve the cross-platform paradigm, one of these is 
%
%Lunatech, having expressed its interest in mobile app development, would like to know which solution, \emph{if any}, suits Lunatechs needs. A study has been set up in order to resolve this question, the results of which are laid out in this thesis.
	
The preliminary research confirmed that Titanium offers cross-platform mobile application development as defined by Lunatech's criteria. The resulting main research tested the viability of Titanium by using it to develop iStager as a case study, and simultaneously by conducting an analysis of the framework.

Although not completely finished due to time restrains, the development of iStager proved that it is possible to build a cross-platform application which is truly native. In addition the realistic nature of the case studies' context proved Titanium can be used for development in a scenario Lunatech might encounter.

The analysis which was performed parallel to the case study resulted in a understanding of the inner working of Titanium, which explained why it provides truly native applications. Next to this the analysis also determined the maturity of the framework using an analytical scheme derived from the OpenSource Maturity Model. 

Based on these statements the following conclusions can be drawn:
\begin{itemize}
\item \emph{Titanium provides the desired extensibility}\\
During the case study it was necessary to extend a feature in Titanium, this was done using a third-party module which augmented Titaniums' SDK with the required feature.

\item \emph{Titanium is a mature open source product}\\
Titanium is a mature framework because the trustworthy elements derived from the OpenSource Maturity Model are present.

\item \emph{Titanium is a well documented open source product}\\
The documentation is up-to-date with the latest version, augmented with code examples, and features an integration with the IDE.

\item \emph{Titanium provides truly native applications}\\
Titanium provides its cross-platform with the \emph{native look-and-feel} trough the use of proxy objects.  A high level API provides JavaScript access to the devices' native features. Each supported platform has its own implementation of the API.
\end{itemize}

\noindent All criteria of the main research are met, consequently its hypothesis may be accepted as a theory:

\begin{shadequote}
Titanium is a viable solution to the cross-platform development of mobile applications while retaining the native look-and-feel%\par\emph{W. de Kraker, 2012}
\end{shadequote}

\noindent Which is a direct answer to the main research question: \emph{"How to develop a cross-platform mobile application while retaining the native look-and-feel?"}

%\paragraaf{Cross-platform Mobile Application Development using Titanium}
%
%
%\paragraaf{Project goals}
\paragraaf{Recommendation}
Titanium provides Lunatech with a viable solution which allows for the cross-platform development of mobile application while retaining the native look-and-feel. Given that both the scope of cross-platform and the definition of native look-and-feel remain unchanged, it is the recommendable to use Titanium as solution to develop mobile applications.

It however in retrospective, it is the opinion of the author that the definition of native look-and-feel was defined to strict. It could have been defined in a broader, more abstract manner, as opposed to coupling it too performance and looks. For example, if the native look-and-feel were to be defined using: 
\begin{itemize}
\item \emph{Distribution}\\
Ability to distribute the application trough its official distribution platform. (Appstore for iOS and Google Play for Android)
\item \emph{Performance}\\
Have direct access to the memory and CPU of the device to allow lightning fast executing of code.
\end{itemize}
\noindent These settings could have changed the outcome of the preliminary research and therefore have resulted in an alternate conclusion.