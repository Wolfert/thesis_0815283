\hoofdstuk{Main research}
%This chapter will define the main research process. 

\paragraaf{Introduction}
%The goal of the main research is to find out if the chosen solution
%The goal of the main research is to get a hands on with the chosen solution. Does it have the features Lunatech wants? Is it suitable for adoption by Lunatech? In other words: "\emph{Is Titanium viable for commerical useage?}".
%
%A solution to the cross-platform paradigm would have to meet a specific set of criteria. 
%
% tijdens het vooronderzoek is naar voren gekomen dat er een bestaande oplossing is die voldoen aan lunatech's criteria. naar aanleiding van dit resultaat 
% op basis van het originele probleem statement




The preliminary research confirmed that there is an existing solution as defined by Lunatech's criteria. In order to determine if this solution is suitable for adoption by Lunatech it is necessary to determine the its viability.
Henceforth primary goal of the main research is to confirm the following hypothesis:
\begin{shadequote}
Titanium provides a viable solution to the cross-platform development of mobile applications%\par\emph{W. de Kraker, 2012}
\end{shadequote}

\noindent The secondary goal of the main research is to analyze how Titanium works and why it provides the desired native \emph{look-and-feel}. The results of this will add to the answer of the main research question.


%Based on the results of the preliminary research was determined that there is an existing solution as defined by Lunatech's criteria. In conclusion of this result


%As discussed in the \emph{Introduction} chapter, Lunatech has stated that it would like to know how to make use of the look-and-feel from native apps with the cross-platform support of web apps. During the preliminary research was determined that there is an existing solution as defined by Lunatech's criteria. Based on this result the following hypothesis has been defined:
%\begin{shadequote}
%Titanium provides a commercially viable solution to the cross-platform development of mobile applications%\par\emph{W. de Kraker, 2012}
%\end{shadequote}





%Therefor the goal of the main research is to define


%so what is viable?
\subparagraaf{Defining viable}
To determine if a cross-platform mobile application development solution is considered viable for usage a set of criteria has been provided by Lunatech.
\begin{itemize}
	\item \emph{Extensibility}\\
	The possibility to extend new features to the framework.
	\item \emph{Maturity}\\ 
	Maturity of framework.
	\item \emph{Documentation}\\
	Coverage, accessibility and up-to-dateness of the documentation.
%	\item \emph{Underlying technique}\\
%	Should provide truly native user interface	
\end{itemize}


{\bf Extensibility} is the possibility to extend Titanium with a new (custom) functionality. 
When the developer wants to implement a custom feature in the solution, the interfacing API should be easy to understand and use

The {\bf maturity} is determined by checking for predefined \emph{trustworthy elements} which are derived from the OpenSource Maturity Model\cite{Wikipedia2011}. The OSMM provides a formal methodology which utilizes a standardized analytical scheme for evaluating the maturity of an open source product. For this research the OSMM scheme has been augmented with elements fit to the specific mobile nature of the framework.

The {\bf documentation} should be publicly accessible, cover all public methods in the most current version of the framework's API.

%it should have all the features implemented which we except to be there. 
%- steady stream of releases.
%- predictable development rate
%- how bugs are handled
%- community support
%
%Accessibility means the documentation is freely available online %and you don't have to look for it. 
%
%These issues come down to the question:
%is the developer spending more time fighting the framework, rather than developing his application?

%how do you want to prove this?
\paragraaf{Method}
In order to determine the viability of the solution it is necessary to perform a case study which originates from a realistic scenario. During the case study the solution will be analyzed on its inner workings. 

The extensibility will be tested by adding a custom or missing feature to the case study application. The following points will be evaluated:
\begin{itemize}
  \setlength{\itemsep}{1pt}
  \setlength{\parskip}{0pt}
  \setlength{\parsep}{0pt}
	\item Requirements for extending a custom feature
	\item Availability of third party extensions
\end{itemize}

\noindent The maturity of the framework will be assessed using an augmented version OpenSource Maturity Model assessment scheme. This scheme consists of a set predefined trustworthy elements:
\begin{itemize}
  \setlength{\itemsep}{1pt}
  \setlength{\parskip}{0pt}
  \setlength{\parsep}{0pt}
\item Product Documentation,
\item Use of Established Mobile Standards,
\item Licenses,
\item Community activity,
\item Number of Commits and Bug Reports
\item Use of code management and versioning tools,
\item Requirements Management,
\item Availability and Use of a product roadmap.
\end{itemize} 
%TODO OSSM twe's verklaren

\noindent Quality of the documentation will be judged on:
\begin{itemize}
  \setlength{\itemsep}{1pt}
  \setlength{\parskip}{0pt}
  \setlength{\parsep}{0pt}
	\item Completeness\\
	Whether or not any missing parts are noticed during the development of the case study.
	\item Up-to-dateness\\
	Determined by the support of the latest version.
	\item Availability\\
	Determined by the location of the documentation and or IDE integration.
\end{itemize}

\begin{tabel}{ p{\Procent{25}} | p{\Procent{25}} | p{\Procent{50}}}{vbx}{Main research method}
\bf{criteria} & \bf{method} & \bf{process}\\
 \hline
Extensibility & Case study & Integrating a custom or missing feature in the case study application\\
Maturity & Analysis & OpenSource Maturity Model assessment\\
Documentation & Case study \& analysis & By usage during the case study development and analyzing it\\
\end{tabel}

\noindent The main research method is divided in two processes, the case study and the analysis. Even though performed parallel, both serve different goals. The global structure is as following:\\

\begin{centering}
\includegraphics[scale=0.5]{images/process.png}\\{Main research process schematic}\\
\end{centering}

% summarize + go ahead for the next two chapters.
%\paragraaf{Summary}
%To answer whether or not the chosen solution is commercially viable a case study will be performed during which the solution will be analyzed on provided criteria.

\hoofdstuk{Case study}

\paragraaf{Introduction}
Mobile application has been developed with Titanium to study its flexibility and features.


\paragraaf{Stager}
In 2011, live music venue WORM hired Lunatech to build \emph{Stager}, a modern web-based resource planning and ticketing application to help manage live music events. Lunatech took the opportunity to use the relatively new Play framework to build a web application with an HTML5 and Java architecture. Stager has broad requirements ranging from high performance and security for the public ticket sales component to high usability for the internal resource planning component that will be used for hours a day by employees and being open to enhancements in the future for new customers. \cite{Lunatech2011}

\subparagraaf{WORM}
WORM is an institute for avantgardistic recreation Rotterdam, consisting of an artistscollective, a podium with a bar and Parallel University (DIY workshops for film, music and media). Born under the stars of punk, Dada, Fluxus, Situationism and futurism WORM is grown into a headstrong organization that the 'Do-It-Yourself' mentality of their ancestors, combined with ultra-pragmatism, love of technique (s) and proper accounting. Worm outputs film, radio, concerts, courses, partys, publications, performances, web projects, installations, workshops and an accumulation of tactile media and internet.WORM focuses (cheerful yet serious) in avantgarde, resource scarcity and opensource. \cite{WORM2012}

\paragraaf{Stager app}
As described in the chapter \emph{Background} Stager is an planning and ticketing application to help manage live music events. In addition to planning and ticketing Stager features an \emph{atomfeed} to publish events. A Stager app would make use of this atomfeed to list any published events on a mobile device.


\paragraaf{Stager application requirements}

List of current and upcoming events
events are downloaded in JSON format from the Stager event atomfeed at /web/feeds/events
events are displayed in a row-based layout
events are linked to their corresponding event detailview
events in the list are sorted by date (asc)
events in the list contain labels with event title, subtitle, date

Detailview of an event
Shows detailed event information of an selected event:
event title, 
subtitle, 
date, 
times (doors open, start, end), 
location details (venue name, street, number, city), 
event content (html rendered text)

Add event to agenda
Prompts the user: "Add event 'x' on date 'y' in agenda?"
Adds a selected event to the mobile devices agenda.
Prompts the user of succes of add action.

Start gps-based navigation towards physical location of event
Prompts the user "Navigate to y?"
Opens map application with address as argument.

View media attached to an event
Media defined as: URL's to event images, videos, websites.

Display in a grid or list, categorize media types.
Each displayed media item is resembled by a tumbnail or icon.
When selected a media item opens to its content in:
images an included webview,
websites the device browser,
for videos the youtube app or the browser( depending on video type \& location).

Share event details to social media
An event detail view will contain a 'share/deel' button.
Prompts the user for platform to share. (twitter/facebook/email)
Default value (editable): "I am attending event x on date y in location x !"

	
(Un)Register device to receive push notifications on new events of interest
Register the device to receive pushed notifications about upcoming events which might be of interest to the user.
Based on Relation.interest model in Stager.



\subparagraaf{Events}



\subparagraaf{Notifications}
\subparagraaf{Tickets}
\subparagraaf{i18n}
\subparagraaf{Mobile payment}
\paragraaf{Used techniques and methodologies}
\subparagraaf{Javascript}
For Titanium Javascript is the only option. Everything that can be written in JavaScript will eventually be written in JavaScript.
\subparagraaf{CommonJS}
\subparagraaf{Playframework}
\subparagraaf{Java}
\subparagraaf{JSON}
%\paragraaf{Titanium modules}
%\paragraaf{Stager service modules}


\paragraaf{Conclusion}
%Samenvatten wat er gedaan is aan de Stager app, welke platforms het draaid, e.d.
\paragraaf{Titanium analysis}
%korte herhaling van wat titanium is, referentie naar existing solutions.

\noindent The primary goal of this paragraph is to determine the maturity of Titanium. As mentioned in \emph{Main research}, section \emph{Defining viable}, the maturity of a framework is determined based on the presence the following of trustworthy elements:

\begin{itemize}
  \setlength{\itemsep}{1pt}
  \setlength{\parskip}{0pt}
  \setlength{\parsep}{0pt}
\item Product Documentation
\item Use of Established Mobile Standards
\item Licenses
\item Community activity
\item Number of Commits and Bug Reports
\item Use of code management and versioning tools
\item Requirements Management
\item Availability and Use of a product roadmap
\end{itemize}

Secondarily this paragraph is to analyze how Titanium works and why it provides the desired native \emph{look-and-feel}. The results of this will add to the answer of the main research question



\subparagraaf{Inner workings}

At runtime a mobile application developed with Titanium consists of three major components:
\begin{itemize}
	\item
	The JavaScript source code
	\item
	A platform-specific implementation of the Titanium API
	\item
	A JavaScript interpreter
\end{itemize}

During runtime the JavaScript source code will be integrated in a native class where it is encoded as a string and compiled. The implementation of the Titanium API done in a platform specific native programming language, Java for Android and Objective-C for iOS. The JavaScript interpreter evaluates the JavaScript code at runtime.


\subparagraaf{Runtime}
At runtime a JavaScript execution environment set up in the native environment, this is where the application source code is evaluated. Injected into JavaScript execution environment are so called \emph{proxy} objects.

\subparagraaf{Proxy objects}
A proxy object is a JavaScript object paired to an object in native code.\cite{Whinnery2012} This means the object exists in both JavaScript and native code. Proxy objects gap the bridge between the native and the JavaScript environment. A global Titanium object in JavaScript exposes access to the proxy objects. 

For example: In the JavaScript code, when a function is called on the global Titanium object to create a native UILabel a proxy object is created.

\begin{minted}[mathescape,
			   label="JavaScript-object",
               linenos,
               numbersep=5pt,
               gobble=0,
               frame=lines,
               framesep=2mm]{js}

var label = Titanium.UI.createTabel({
   text: "Lorem impsum",
   top: 10,
   left: 10,
   width: 100,
   height: 20
});
\end{minted}


In iOS the proxy button object:

\begin{minted}[linenos,
				label="Native-object",
				samepage,
				tabsize=2,
				xleftmargin=0cm,
               numbersep=5pt,
               frame=lines,
               framesep=2mm]{objc}
-(UILabel*)label
{
    if (label==nil)
    {
        label = [[UILabel alloc] initWithFrame:CGRectZero];
        label.backgroundColor = [UIColor clearColor];
        label.numberOfLines = 0;
        [self addSubview:label];
    }
    return label;
}
\end{minted}

\subparagraaf{JavaScript}


As mentioned above, Titanium uses JavaScript for cross-platform code compatibility. JavaScript is a prototype based scripting language.\cite{Crockford2008}. JavaScript is a logical choice because there are JavaScript Interpreters available for most platforms. This includes the targeted mobile platforms:

V8 is the default JavaScript Interpreter for Android but Rhino is also supported. V8 is has a better performance than Rhino because it is directly integrated to the NDK\footnote{Native Development Kit}. This means the code does not have to run trough the JVM\footnote{Java Virtual Machine}. The performance gain can exceed over 200\% processing time when parsing a JSON object.\cite{Lukasavage2011}
For iOS JavaScriptCore is the chosen interpreter.

These interpreters support the CommonJS specification.


\subparagraaf{CommonJS}
Although JavaScript is a powerful object oriented language with dynamic interpreters available on most platforms, the official specification only defines APIs for objects that are designed for developing browser-based applications.
The CommonJS API will fill that gap by defining APIs that handle many common application needs, ultimately providing a standard library as rich as those of Python, Ruby and Java.\cite{CommonJS2012a} The idea behind this is that the developer can use JavaScript to develop fullgrown applications rather than being limited to the browserenvironment.

Titanium implements the CommonJS module specification.


\subparagraaf{Modules}
JavaScript does not have any name spacing by default, this means that all objects and function live in a global scope. As a consequence name conflicts on functions and variables can cause runtime errors. 
To prevent pollution of the global scope the developer can make use of CommonJS modules.
CommonJS modules solve JavaScript scope issues by making sure each module is executed in its own namespace.\cite{CommonJS2012b} In CommonJS modules export the variable which are to be exposed to other modules. Other modules are explicitly imported where they are required. As a consequence, modules can prevent variables from clogging up the global namespace.
~\\
A sample module:


\begin{minted}[mathescape,
         label="eventCell.js",
               linenos,
               numbersep=5pt,
               gobble=0,
               frame=lines,
               framesep=2mm]{js}
function eventCell(Event, delegate) {
  // cell code removed
  return this.cell;
};
module.exports = eventCell;
\end{minted}

Is used like:
\begin{minted}[mathescape,
         label="eventCell.js",
               linenos,
               numbersep=5pt,
               gobble=0,
               frame=lines,
               framesep=2mm]{js}
  var messageCell = require('stager/tableview/messageCell');
  tableview.appendRow(new messageCell(message, that));
\end{minted}

\subparagraaf{Maturity}
As mentioned the maturity of a framework is determined based on the presence of the predefined of trustworthy elements.

\begin{description}
\item [Product Documentation] is found at the official website as well as a simplified version which included on the code completion.
The documentation is updated either on the day of a new release or the day after.\cite{Inc2012b} A simplified version of the documentation is integrated into TitaniumStudio, a feature named \emph{'dynamic help'} automatically looks up the documentation belonging to the a selected method or object.
Many objects in the documentation are supplemented with an code example.
\item [Use of Established Mobile Standards] 
On the surface an fully JavaScript API, Titanium also has support for CommonJS module specifications. 
\item [Licenses] Appcelerator offers four plans from which the first one is free, the other three provide different levels of support and include the use of cloud services. The commercial plans' pricing is per application. Additionally Appcelerator offers services for app analytics or stand alone cloud services. A drawback is that prices are not publicly listed.
\item [Community activity] Titaniums community exists of mobile developers whom use Titanium to develop their mobile applications. Some of these have knowledge of native programming on the supported platforms. It happens frequently that they publish features they implement because it was missing from Titanium. 
\item [Number of Commits and Bug Reports] Since the beta there have been 15,991 commits. %https://github.com/appcelerator/titanium_mobile/branches
On Titaniums' Jira page there have been 8784 issues reported of which 2601 are open.
Issues: 398 created and 366 resolved
%TODO: conclusie formuleren.
\item [Use of code management and versioning tools] Titaniums' code base is maintained in a public Github account.%TODO github verklaren account. For 
\item [Requirements Management] Requirements and features are managed using Jira tickets.
\item [Availability and Use of a product roadmap] Not publicly available,\\
\texttt{http://developer.appcelerator.com/doc/mobile/roadmap}\\
"We apologize but at this time we're not publishing our roadmap."
\end{description}

\paragraaf{Results}
This paragraph summarizes the results of the main research.

\subparagraaf{Extensibility}
It is possible to extend new functionality in Titanium. This is done at two levels:
\begin{itemize}
	\item Project based
	\item SDK based
\end{itemize}
The Project based extending of features is done by placing the module in the Titaniums installation directory. SDK based means that the module is built inline with the Titanium's source code, from there the module augments Titanium's SDK.  Extending Titanium's features at SDK level requires the entire framework to be rebuilt and compiled which can take up to 50 minutes. When a module is added on project base it only requires a restart from the framework, this process might take up to 2 minutes. Another downside is that a custom version of Titaniums SDK will have to be built and compiled for every new release. Therefore it is preferable to extend new functionality on project base.

\subparagraaf{Maturity}
The following diagram is based on observations done during the analysis.
\begin{tabel}{ >\R p{\Procent{60}} | p{\Procent{20}} }{vbx}{Maturity level}
\bf{Thrustworthy element} & \bf{Score}\\
 \hline
Product Documentation& +++\\
Use of Established Mobile Standards&++\\
Licenses&++\\
Community activity&++\\
Number of Commits and Bug Reports &+\\
Use of code management and versioning tool &+++\\
Requirements Management&+\\
Availability and Use of a product roadmap\footnote{In use but not publicly available, http://developer.appcelerator.com/doc/mobile/roadmap}&-+\\
\end{tabel}


\subparagraaf{Documentation}
During the case study the documentation of Titanium was used daily for a period of 8 weeks. The following observations were made: The documentation is located at a central web resource and in addition has been integrated in Titanium's IDE. All methods and functions which were needed during the development of the case study were found in the documentation. During the development of the case study a new version of Titanium was release, the documentation was update consequently. Each described function has a simple overview of which platforms it supports and in which SDK version it is included. As a result of these observations the following score's are met:

\begin{tabel}{ >\R p{\Procent{60}} | p{\Procent{20}} }{vbx}{Maturity level}
\bf{Criteria} & \bf{Score}\\
 \hline
Completeness& +++\\
Up-to-dateness&+++\\
Availability&+++\\
\end{tabel}

