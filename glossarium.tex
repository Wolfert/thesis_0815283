\hoofdstuk{Appendix III - Glossarium}


\begin{description}
\item[Android] is a open source mobile operating system, developed by the Open Handset Alliance, led by Google and other companies.\cite{Inc.2012}

\item[Apple Appstore] is the offical app distribution platform for OS X and iOS, by Apple Inc.

\item[Atom feed] is a webfeed published in standardised format.

\item[Bluetooth] is a wireless communication technique.

\item[DOM specifications] DOM stands for Document Object Model, the specification describes a standard for representing and interacting with an object.

\item[GPS] stands for Global Positioning System, a technique which uses orbiting sattalites to determine a position on earth.

\item[i18n] is an a abbreviated numeronym of \emph{internationalization}, where 18 stands for the number of letters between the first i and last n in internationalization.

\item[IDE] stands for Intergrated Development Environment, is a tool which provides an enviroment to aid a developer during the coding process, an example of an IDE is Eclipse.

\item[iOS] is a proprietary mobile operating system, developed by Apple Inc. It was originally released in 2007 for the iPhone and iPod Touch. iOS also became the main operating system of the iPad and Apple TV.\cite{Sylvain2012}

\item[Hardware acceleration] is a techique used to when functions are performed by hardware rather than by the running software. An example of this is the pixelshifting, instead of redrawing, directly via GPU in iOS native apps.

\item[HTML] stands for Hyper Text Markup Language, a language used for the markup of websites.

\item[JavaScript] is objectoriented scripting language of prototypal nature.

\item[Prototype-based] is a object-oriented programming style in which behaviour is defined by objects ranther than classes. An object can contain both behaviour and data. The prototype of an object can be reused to inherit behaviour.

\item[R] is a program used for statistical and mathematical calculations.

\item[SDK] stands for Software Development Kit, contains a collections of libraries which provide the framework on which an application can be developed.

\item[Sandbox environment] is an environment seperated from the environment it is situated in.

\item[XML] stands for Exensible Markup Language, is used to markup structured data in plaintext format.


\end{description}