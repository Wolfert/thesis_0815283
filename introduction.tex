\hoofdstuk{Introduction}

\ifdraftmode
Currently.. mobile devices rule the world. However not all smartphones run the same operatingsystem and thus apps have to be build platform specific.  This poses an obstacle for developers because it takes an n codebases for n number of supported platforms which brings high development and maintance costs. Ideally one would build from a codebase once and have the app run on every platform. 

On todays market there are several frameworks which offer this possibility. Most of which use a browser based environment to run their apps. This demands the userinterface to be build up in HTML5/CSS which don't feel very native as UI elements have to be parsed and rendered rather than generated have enjoy hardware acceleration. There are however some that generate native code to each platform, and accidentaly may create SkyNet.

because they statisfy to need to have information available at any time, from any where. To help the user gain this acces apps are build purposed to perform a single or set of key tasks. such as retreiving news, email, sharing photo's. blaat\cite{Ni2006}
\fi
\paragraaf{Problem statement}



Lunatech has demand for the development of cross-platform mobile applications. Currently\footnote{Note: when mentioning the word 'current', it refers to the old situation as the process to get to the actual current situation is being illustrated} these applications are been developed using webtechnologies such as HTML5 and Javascript. A mobile application developed this way is refered to as webapp because it runs in a browserbased environment and is often hosted at a webserver rather than downloaded to the device itself.

The problem with webapps is that they lack in user experience. This is mainly due manner in which user interface components are build in HTML. Every platform has its own set of recognizable elements, but these cannot be accessed from within the browser environment. As a result of this the app will feel unearthly to the user because it's style doesn't match the rest of the platform. It tries to look and feels native, but never gets arround the fact that it's a webapp.

The direct alternative to webapps are native apps, native are writting using technologies proprietary to each platform, hench the term 'native'. What these applications lose in terms of cross-platform support they make up in terms of user experience.  A native app has acces to all the platforms propietary libraries and can rely on the user interface elements provided through these libraries.

Lunatech would like to know how to make use of the look-and-feel from native apps with the cross-platform support of webapps.

\paragraaf{Research questions}

Main research question:
\begin{itemize}
\item \emph{How to develop a cross-platform mobile application while retaining the native look-and-feel?}
\end{itemize}

\noindent Sub research questions:
\begin{itemize}
\item \emph{How is the native look-and-feel defined?}
\item \emph{Which solutions to cross-platform mobile application development currently exist?}
\item \emph{Which of these solutions offer the defined native look-and-feel?}
\item \emph{Which mobile platforms should be targetted for cross-platform mobile application development?}
\end{itemize}