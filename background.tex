\hoofdstuk{Background}
\paragraaf{Lunatech Research B.V}
Lunatech provides application development services, completely based on open-source web and Java technologies and open standards. They are early adopters of new technology, and use cutting-edge frameworks and tools. To stay up-to-date, their developers have the opportunity to research, try new technologies and contribute to open-source projects. The company is dominated by software developers. Everyone (except the director) writes code, on top of which some staff have a secondary management role, and the staff who will deliver a project interact with the customer directly.

\paragraaf{Rotterdam University of Applied Sciences (Hogeschool Rotterdam)}
Rotterdam University is one of the major Universities of Applied Sciences in the Netherlands. Currently almost 30,000 students are working on their professional future at the university.
The university is divided into eleven schools, offering more than 80 graduate and undergraduate programmes in seven fields: art, technology, media and information technology, health, behaviour and society, engineering, education, and of course, business.\cite{HogeschoolRotterdam2012}

\paragraaf{Stager}
In 2011, live music venue WORM - Instituut voor Avantgardistische Recreatie hired Lunatech to build \emph{Stager}, a modern web-based resource planning and ticketing application to help manage live music events. Lunatech took the opportunity to use the relatively new Play framework to build a web application with an HTML5 and Java architecture. Stager has broad requirements ranging from high performance and security for the public ticket sales component to high usability for the internal resource planning component that will be used for hours a day by employees and being open to enhancements in the future for new customers. \cite{Lunatech2011} %todo: bron toevoegen

\subparagraaf{WORM}
WORM is een instituut voor avantgardistische recreatie te Rotterdam, bestaande uit een kunstenaarscollectief, een podium met winkel en een Parallelle Universiteit (DIY-werkplaatsen voor film, muziek en media). Geboren onder de sterren van punk, dada, fluxus, situationisme en futurisme is WORM uitgegroeid tot een eigengereide organisatie die de ‘Do-It-Yourself’ mentaliteit van hun voorouders combineert met ultra-pragmatisme, liefde voor techniek(en) en goede boekhouding. De output van WORM is film, radio, concerten, cursussen, partys, publicaties, performances, web-projecten, installaties, workshops en een opeenhoping van tactiele media en internet.WORM focust zich (blijmoedig en toch serieus) op avantgarde, middelenschaarste en opensource.\#todo: translate \cite{WORM2012} %TODO bron toevoegen