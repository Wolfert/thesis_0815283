\hoofdstuk{Preliminary research}


The goal of the preliminary research is to determine whether or not an existing solution offers cross-platform mobile application development as defined by Lunatechs' criteria.


\paragraaf{Introduction}
In todays industry there exist several cross-platform mobile application development frameworks which offer a solution to cross-platform problem. All of these frameworks provide a solution for crossing the bridge between platforms. In order determine which one should be adopted by Lunatech for mobile development the following criteria have been determined for comparisson:

\begin{enumerate}
\item \emph{Platform support}\\
Which platforms and their versions are supported by the framework.
\item \emph{Native UI support}\\
Whether or not native user interface elements are supported for each supported platform.
\item \emph{Programming language}\\
Which programming language is used to develop using the framework.
\item \emph{IDE} (Intergrated Development Environment)\\
Which IDE can be used to develop using with the framework.
\item \emph{License type}\\
Which license types are available.
\item \emph{Application type}\\
Which type of mobile application is produced using this framework.
\end{enumerate}

The cross-platform criterium is based on Lunatechs requirement to build mobile applications for the operating systems have at least a 10 percent marketshare in the European continent. Second comes the support for native user interface elements. Together these criteria form the essence of the main research question: \emph{"How to develop a cross-platform mobile application while retaining the native look-and-feel?"}
The remaining criteria are of secondary importance, they will provide more detailed means to compare the frameworks which offer native user interface support.

\paragraaf{Framework requirements} 

A cross-platform mobile application development framework has to:
\begin{itemize}
\item Support cross-platform mobile application development.\\
Build an application which runs on multiple platforms, originating from a single codebase.
\item Supported platforms should be at minium iOS and Android.\\
As decided in Chapter x. Definitions - \emph{Mobile platforms}
\item Be able to offer the native look-and-feel.\\
As defined in Chapter x. Definitions - \emph{Defining native}
\end{itemize}

\paragraaf{Method}
% listing van bestaande frameworks.
% toepassen criterea.
% testwijze (benchmak app)
% beste kiezen om mee verder te gaan

\subparagraaf{Selection}

There are over 30 frameworks which offer cross-platform development of mobile applications\cite{Wikipedia2012}. These frameworks have been added to an initial list \footnote{see appendix: \emph{Existing solutions}} This intial list contains only frameworks which adhere to the requirement of supporting cross-platform mobile application development.

To determine which frameworks should be included in the comparison we'll take a closer look at each and filter out those who don't adhere to criteria of supporting native user elements. This should leave us with less than x frameworks to compare. which is the goal because the timeframe of the internship doesn't allow for more. 




\subparagraaf{Evaluation process}
Estimated is that it takes 4 days to do experiments with a framework, this is based on N+1 in which N is the number of days it took to develop the benchmark app + 1 to familiarize with the framework. If in those 4 days I'm unable to complete the benchmark app, it is still a result. The experiment consumes a timespan of 32 hours per framework and consist of the following activities:

\begin{itemize}
	\item Install framework
	\item Familiarize with how it works
	\item Rewrite the benchmark app in it
	\item Noting down results and documenting the experience
\end{itemize}

The remaining framework will be Evaluated on available of documentation, licensing, community, and flexibilityy. The research itself is included as an appendix.

\paragraaf{Results}

From 30 existing solutions a total of 4 frameworks have been selected to get evaluated:
\begin{itemize}
\item Titanium
\item RhoMobile
\item MoSync
\item Worklight
\end{itemize}


todo.
