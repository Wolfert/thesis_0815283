\paragraaf{Titanium analysis}
%korte herhaling van wat titanium is, referentie naar existing solutions.

\noindent The primary goal of this paragraph is to determine the maturity of Titanium. As mentioned in \emph{Main research}, section \emph{Defining viable}, the maturity of a framework is determined based on the presence the following of trustworthy elements:

\begin{itemize}
  \setlength{\itemsep}{1pt}
  \setlength{\parskip}{0pt}
  \setlength{\parsep}{0pt}
\item Product Documentation
\item Use of Established Mobile Standards
\item Licenses
\item Community activity
\item Number of Commits and Bug Reports
\item Use of code management and versioning tools
\item Requirements Management
\item Availability and Use of a product roadmap
\end{itemize}

Secondarily this paragraph is to analyze how Titanium works and why it provides the desired native \emph{look-and-feel}. The results of this will add to the answer of the main research question



\subparagraaf{Inner workings}

At runtime a mobile application developed with Titanium consists of three major components:
\begin{itemize}
	\item
	The JavaScript source code
	\item
	A platform-specific implementation of the Titanium API
	\item
	A JavaScript interpreter
\end{itemize}

During runtime the JavaScript source code will be integrated in a native class where it is encoded as a string and compiled. The implementation of the Titanium API done in a platform specific native programming language, Java for Android and Objective-C for iOS. The JavaScript interpreter evaluates the JavaScript code at runtime.


\subparagraaf{Runtime}
At runtime a JavaScript execution environment set up in the native environment, this is where the application source code is evaluated. Injected into JavaScript execution environment are so called \emph{proxy} objects.

\subparagraaf{Proxy objects}
A proxy object is a JavaScript object paired to an object in native code.\cite{Whinnery2012} This means the object exists in both JavaScript and native code. Proxy objects gap the bridge between the native and the JavaScript environment. A global Titanium object in JavaScript exposes access to the proxy objects. 

For example: In the JavaScript code, when a function is called on the global Titanium object to create a native UILabel a proxy object is created.

\begin{minted}[mathescape,
			   label="JavaScript-object",
               linenos,
               numbersep=5pt,
               gobble=0,
               frame=lines,
               framesep=2mm]{js}

var label = Titanium.UI.createTabel({
   text: "Lorem impsum",
   top: 10,
   left: 10,
   width: 100,
   height: 20
});
\end{minted}


In iOS the proxy button object:

\begin{minted}[linenos,
				label="Native-object",
				samepage,
				tabsize=2,
				xleftmargin=0cm,
               numbersep=5pt,
               frame=lines,
               framesep=2mm]{objc}
-(UILabel*)label
{
    if (label==nil)
    {
        label = [[UILabel alloc] initWithFrame:CGRectZero];
        label.backgroundColor = [UIColor clearColor];
        label.numberOfLines = 0;
        [self addSubview:label];
    }
    return label;
}
\end{minted}

\subparagraaf{JavaScript}


As mentioned above, Titanium uses JavaScript for cross-platform code compatibility. JavaScript is a prototype based scripting language.\cite{Crockford2008}. JavaScript is a logical choice because there are JavaScript Interpreters available for most platforms. This includes the targeted mobile platforms:

V8 is the default JavaScript Interpreter for Android but Rhino is also supported. V8 is has a better performance than Rhino because it is directly integrated to the NDK\footnote{Native Development Kit}. This means the code does not have to run trough the JVM\footnote{Java Virtual Machine}. The performance gain can exceed over 200\% processing time when parsing a JSON object.\cite{Lukasavage2011}
For iOS JavaScriptCore is the chosen interpreter.

These interpreters support the CommonJS specification.


\subparagraaf{CommonJS}
Although JavaScript is a powerful object oriented language with dynamic interpreters available on most platforms, the official specification only defines APIs for objects that are designed for developing browser-based applications.
The CommonJS API will fill that gap by defining APIs that handle many common application needs, ultimately providing a standard library as rich as those of Python, Ruby and Java.\cite{CommonJS2012a} The idea behind this is that the developer can use JavaScript to develop fullgrown applications rather than being limited to the browserenvironment.

Titanium implements the CommonJS module specification.


\subparagraaf{Modules}
JavaScript does not have any name spacing by default, this means that all objects and function live in a global scope. As a consequence name conflicts on functions and variables can cause runtime errors. 
To prevent pollution of the global scope the developer can make use of CommonJS modules.
CommonJS modules solve JavaScript scope issues by making sure each module is executed in its own namespace.\cite{CommonJS2012b} In CommonJS modules export the variable which are to be exposed to other modules. Other modules are explicitly imported where they are required. As a consequence, modules can prevent variables from clogging up the global namespace.
~\\
A sample module:


\begin{minted}[mathescape,
         label="eventCell.js",
               linenos,
               numbersep=5pt,
               gobble=0,
               frame=lines,
               framesep=2mm]{js}
function eventCell(Event, delegate) {
  // cell code removed
  return this.cell;
};
module.exports = eventCell;
\end{minted}

Is used like:
\begin{minted}[mathescape,
         label="eventCell.js",
               linenos,
               numbersep=5pt,
               gobble=0,
               frame=lines,
               framesep=2mm]{js}
  var messageCell = require('stager/tableview/messageCell');
  tableview.appendRow(new messageCell(message, that));
\end{minted}

\subparagraaf{Maturity}
As mentioned the maturity of a framework is determined based on the presence of the predefined of trustworthy elements.

\begin{description}
\item [Product Documentation] is found at the official website as well as a simplified version which included on the code completion.
The documentation is updated either on the day of a new release or the day after.\cite{Inc2012b} A simplified version of the documentation is integrated into TitaniumStudio, a feature named \emph{'dynamic help'} automatically looks up the documentation belonging to the a selected method or object.
Many objects in the documentation are supplemented with an code example.
\item [Use of Established Mobile Standards] 
On the surface an fully JavaScript API, Titanium also has support for CommonJS module specifications. 
\item [Licenses] Appcelerator offers four plans from which the first one is free, the other three provide different levels of support and include the use of cloud services. The commercial plans' pricing is per application. Additionally Appcelerator offers services for app analytics or stand alone cloud services. A drawback is that prices are not publicly listed.
\item [Community activity] Titaniums community exists of mobile developers whom use Titanium to develop their mobile applications. Some of these have knowledge of native programming on the supported platforms. It happens frequently that they publish features they implement because it was missing from Titanium. 
\item [Number of Commits and Bug Reports] Since the beta there have been 15,991 commits. %https://github.com/appcelerator/titanium_mobile/branches
On Titaniums' Jira page there have been 8784 issues reported of which 2601 are open.
Issues: 398 created and 366 resolved
%TODO: conclusie formuleren.
\item [Use of code management and versioning tools] Titaniums' code base is maintained in a public Github account.%TODO github verklaren account. For 
\item [Requirements Management] Requirements and features are managed using Jira tickets.
\item [Availability and Use of a product roadmap] Not publicly available,\\
\texttt{http://developer.appcelerator.com/doc/mobile/roadmap}\\
"We apologize but at this time we're not publishing our roadmap."
\end{description}